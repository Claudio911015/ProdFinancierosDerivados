\documentclass{article}
\title{Tarea Exámen \\ Productos Financieros Derivados}
\author{Universidad Marista}

\date{\today}

\begin{document}

\maketitle


\section{Instrucciones}
El alumno deberá de hallar los factores de descuento 
correspondientes a los vencimientos de los instrumentos 
mencionados a continuación implementando un algoritmo de BootStrapping considerando las
siguientes características:
\begin{itemize}
\item Convención de conteo de días ACT/365
\item Convención de las tasas presentadas: Linear
\item No existen días inhábiles.
\item La curva se interpola de forma log- linear
\item Tasas Libor y swap expresadas en porcentaje
\item instrumentos:
    \begin{enumerate}
        \item Libor 3M 0.2
        \item Libor 6M 0.4
        \item Libor 9M 0.6
        \item Futuro Libor 3M (6M, 12M) 99.2
        \item Futuro Libor 3M (9M, 15M) 99.56
        \item Swap Libor 3M/6M 2Y 0.75
        \item Swap Libor 3M/6M 3Y 0.79
        \item Swap Libor 3M/6M 4Y 0.82
        \item Swap Libor 3M/6M 5Y 0.83
        \item Swap Libor 3M/6M 6Y 0.834
        \item Swap Libor 3M/6M 9Y 0.841
        \item Swap Libor 3M/6M 10Y 0.846
        \item Swap Libor 3M/6M 15Y 0.8502
        \item Swap Libor 3M/6M 20Y 0.9
    \end{enumerate}

\item La frecuencia de pago de los Swaps es de 6 Meses.


\end{itemize}


\end{document}