\documentclass{article}
\title{Examen 1 \\ Productos Financieros Derivados}
\author{Universidad Marista}

\date{\today}

\begin{document}

\maketitle


\section{Instrucciones}
Conteste las siguientes preguntas y deberá enviar las respuestas correspondientes
al correo claudio.cuevas@umarista.edu.mx a más tardar el día 13 de Septiembre. Para el conteo de días, utilizar
la convención 30 / 360 en caso de no especificarse una convención particular. Asuma que las tasas libres de riesgo son las tasas de referencia (LIBOR, TIIE, EURIBOR).


\section*{} 

\begin{enumerate}
\item Definición de un contrat forward.
\item ¿Cómo se define el precio forward asociado a un contrato forward?
\item Grafique el payoff correspondiente a un contrato forward a un plazo de 3 meses cuyo Strike es de \$ 20.00.
\item Suponga que cierra un forwatd largo sobre el precio del oro, el cuál está en \$ 1800.00 USD la onza. El 
    contrato forward establece un precio Strike sobre el Oro de \$2000.00 USD por Onza. Considere además
    que el costo de almacenaje del oro es de \$50.00 USD al mes y la tasa libre de riesgo es del 2\% anual contínua.
    Determine lo siguiente:
    \begin{itemize}
        \item El valor del contrato forward.
        \item El precio forward del contrato.
    \end{itemize}

\item Un gestor de fondos de pensiones sabe que dentro de tres meses va a
recibir \$10,000,000.00 de euros procedentes de nuevas aportaciones. Su
objetivo es garantizar una rentabilidad para estos fondos durante 6 meses
en los que sabe que no los va a necesitar para afrontar pagos. El gestor de
pensiones tiene expectativas de bajadas futuras de los tasas de interés.
Sabiendo que en la actualidad los tasas de interés son los siguientes:
EURIBOR(3 meses): 1,25\% y EURIBOR(9 meses): 1,95\%. ¿Qué puede
hacer para cubrir la bajada de tasas? Seleccione una de las siguientes
 alternativas y justifique su respuesta.

\begin{enumerate}
    \item Comprar un FRA(3,9) al 2,30\%.
    \item Comprar un FRA(3,9) al 1,75\%.
    \item Vender un FRA(3,9) al 2,30\%.
    \item Vender un FRA(3,9) al 1,75\%.
\end{enumerate}

\item ¿Cómo se verá afectado el comprador de un Forward sobre aluminio por
un aumento de su precio al vencimiento del contrato?

\begin{enumerate}
    \item Tendrá pérdidas porque ha comprado anticipadamente un activo a un
    precio inferior al de mercado.
    \item Tendrá beneficios porque ha comprado anticipadamente un activo a un
    precio inferior al de mercado.
    \item Tendrá pérdidas si liquida la operación antes del vencimiento del
    contrato.
    \item No le afecta.
\end{enumerate}

\item Considere un contrato forward sobre la acción de IBM. 
      Asumir que dicha acción cotiza actualmente en \$100 dólares.
      IBM paga de forma trimestral un dividendo de \$1 dólar por acción durante
      los siguientes 5 años. La tasa anual compuesta es de 2\% constante para
      todos los vencimientos. Calcule lo siguiente:
      \begin{itemize}
        \item El valor del contrato forward considerando un strike de \$100 dólares.
        \item El Precio forward del instrumento.
        \item Supongamos que el precio forward está cotizando en \$103 dólares, diseñe
            una estrategia de arbitraje tal que tengamos sin ningún riesgo una ganancia asegurada.
            \item Supongamos que el precio forward está cotizando en \$99 dólares, diseñe
            una estrategia de arbitraje tal que tengamos sin ningún riesgo una ganancia asegurada.
      \end{itemize}


\end{enumerate}



\end{document}